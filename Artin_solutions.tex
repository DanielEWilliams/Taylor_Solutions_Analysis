% Place holder file for solutions to 
% Arten's Algebra

\documentclass[12pt]{book}

\usepackage[margin=1in]{geometry}
\usepackage{amsmath,amsthm,amssymb,hyperref}

\newcommand{\N}{\mathbb{N}}
\newcommand{\Z}{\mathbb{Z}}
\newcommand{\R}{\mathbb{R}}
\newcommand{\Q}{\mathbb{Q}}

\hypersetup{
    bookmarks=true,
    unicode=true,
    pdftitle={Michael Senter}
}

\newenvironment{theorem}[2][Theorem]{\begin{trivlist}
\item[\hskip \labelsep {\bfseries #1}\hskip \labelsep {\bfseries #2.}]}{\end{trivlist}}
\newenvironment{lemma}[2][Lemma]{\begin{trivlist}
\item[\hskip \labelsep {\bfseries #1}\hskip \labelsep {\bfseries #2.}]}{\end{trivlist}}
\newenvironment{exercise}[2][Exercise]{\begin{trivlist}
\item[\hskip \labelsep {\bfseries #1}\hskip \labelsep {\bfseries #2.}]}{\end{trivlist}}
\newenvironment{problem}[2][Problem]{\begin{trivlist}
\item[\hskip \labelsep {\bfseries #1}\hskip \labelsep {\bfseries #2.}]}{\end{trivlist}}
\newenvironment{question}[2][Question]{\begin{trivlist}
\item[\hskip \labelsep {\bfseries #1}\hskip \labelsep {\bfseries #2.}]}{\end{trivlist}}
 
\newenvironment{corollary}[2][Corollary]{\begin{trivlist}
\item[\hskip \labelsep {\bfseries #1}\hskip \labelsep {\bfseries #2.}]}{\end{trivlist}}



\begin{document}

\title{Solutions Manual for Michael Arten's \\ \emph{Algebra}}%replace X with the appropriate number
\author{Michael Senter}

\maketitle


\chapter{Matrix Operations}

\section{Basic Operations}
\section{Row Reduction}
\section{Determinants}
\section{Permutation Matrices}
\section{Cramer's Rule}




\chapter{Groups}

\section{The Definition of a Group}
\section{Subgroups}
\section{Isomorphisms}
\section{Homomorphisms}
\section{Equivalence Relations and Partitions}
\section{Cosets}
\section{Restriction of a Homomorphism to a Subgroup}
\section{Products of Groups}
\section{Modular Arithmetic}
\section{Quotient of Groups}




\chapter{Vector Spaces}

\section{Real Vector Spaces}
\section{Abstract Fields}
\section{Bases and Dimension}
\section{Computation with Bases}
\section{Infinite-Dimensional Spaces}
\section{Direct Sums}




\chapter{Linear Transformations}

\section{The Dimension Formula}
\section{The Matrix of a Linear Transformation}
\section{Linear Operators and Eigenvectors}
\section{The Characteristic Polynomial}
\section{Orthogonal Matrices and Rotations}
\section{Diagonalization}
\section{Systems of Differential Equations}
\section{The Matrix Exponential}




\chapter{Symmetry}

\section{Symmetry of Plane Figures}
\section{The Group of Motions of the Plane}
\section{Finite Groups of Motions}
\section{Discrete Groups of Motions}
\section{Abstract Symmetry: Group operations}
\section{The Operation on Cosets}
\section{The Counting Formula}
\section{Permutation Representation}
\section{Finite Subgroups of the Rotation Group}




\chapter{More Group Theory}

\section{The Operations of a Group on Itself}
\section{The Class Equation of the Icosahedral Group}
\section{Operations on Subsets}
\section{The Sylow Theorem}
\section{The Groups of Order 12}
\section{Computation in the Symmetric Group}
\section{The Free Group}
\section{Generators and Relations}
\section{The Todd-Coxeter Algorithm}



\chapter{Bilinear Forms}




\chapter{Linear Groups}




\chapter{Group Representations}




\chapter{Rings}




\chapter{Factorization}




\chapter{Modules}





\chapter{Fields}




\chapter{Galois Theory}








\end{document}
