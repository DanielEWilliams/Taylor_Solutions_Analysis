% --------------------------------------------------------------
% This is all preamble stuff that you don't have to worry about.
% Head down to where it says "Start here"
% --------------------------------------------------------------

\documentclass[12pt]{article}

\usepackage[margin=1in]{geometry}
\usepackage{amsmath,amsthm,amssymb}

\newcommand{\N}{\mathbb{N}}
\newcommand{\Z}{\mathbb{Z}}
\newcommand{\R}{\mathbb{R}}
\newcommand{\Q}{\mathbb{Q}}

\newenvironment{theorem}[2][Theorem]{\begin{trivlist}
\item[\hskip \labelsep {\bfseries #1}\hskip \labelsep {\bfseries #2.}]}{\end{trivlist}}
\newenvironment{lemma}[2][Lemma]{\begin{trivlist}
\item[\hskip \labelsep {\bfseries #1}\hskip \labelsep {\bfseries #2.}]}{\end{trivlist}}
\newenvironment{exercise}[2][Exercise]{\begin{trivlist}
\item[\hskip \labelsep {\bfseries #1}\hskip \labelsep {\bfseries #2.}]}{\end{trivlist}}
\newenvironment{problem}[2][Problem]{\begin{trivlist}
\item[\hskip \labelsep {\bfseries #1}\hskip \labelsep {\bfseries #2.}]}{\end{trivlist}}
\newenvironment{question}[2][Question]{\begin{trivlist}
\item[\hskip \labelsep {\bfseries #1}\hskip \labelsep {\bfseries #2.}]}{\end{trivlist}}
 
\newenvironment{corollary}[2][Corollary]{\begin{trivlist}
\item[\hskip \labelsep {\bfseries #1}\hskip \labelsep {\bfseries #2.}]}{\end{trivlist}}

\begin{document}

% --------------------------------------------------------------
%                         Start here
% --------------------------------------------------------------

\title{Solutions Manual for Joseph Taylor's \\ \emph{Foundations of Analysis}}%replace X with the appropriate number
\author{Michael Senter}
\date{}

\maketitle

\begin{center}
\textsc{\Large Section 1.1}
\end{center}

\begin{exercise}{1.1.1} %You can use theorem, exercise, problem, or question here.  Modify x.yz to be whatever number you are proving
If $a, b \in \mathbb{R}$ and $a < b$, give a description in set theory notation for each of the intervals $(a, b)$, $[a, b]$, $[a, b)$, and $(a, b]$ (see Example 1.1.1).
\begin{align*}
	(a,b) &= \{x \in \mathbb{R} : a < x < b\} \\
	[a,b] &= \{x \in \mathbb{R} : a \leq x \leq b\} \\
	[a,b) &= \{x \in \mathbb{R} : a \leq x < b\} \\
	(a,b] &= \{x \in \mathbb{R} : a < x \leq b\}.
\end{align*}
\end{exercise}

\begin{theorem}{1.1.2}
If $A$, $B$, and $C$ are sets, then $A  \cap (B \cup C) = (A \cap B) \cup (A \cap C)$.
\end{theorem}
\begin{proof}
If  $x \in A  \cap (B \cup C)$, then $x \in A$ and $x \in (B \cup C)$. Thus, either $x \in B$ or $x \in C$. Thus, $x \in A \cap B$ or in $x \in A \cap C$. Thus, if $x \in A  \cap (B \cup C)$, then $x \in (A \cap B) \cup (A \cap C)$. \\
If an $x \in (A \cap B) \cup (A \cap C)$, then either $x \in (A \cap B)$ or $x \in (A \cap C)$. This means that surely $x \in A$, and also that $x \in B \cup C$. Hence, if $x \in (A \cap B) \cup (A \cap C)$, then $x \in A  \cap (B \cup C)$.\\ 
Therefore, $A  \cap (B \cup C) = (A \cap B) \cup (A \cap C)$.
\end{proof}


\begin{exercise}{1.1.3}
\emph{Solved in Class}
\end{exercise}


\begin{question}{1.1.5}
What is the intersection of all the closed intervals containing the open interval $(0, 1)$? Justify your answer. \\

Let $\mathcal{A}$ denote the set of all sets such that $(0,1) \in \mathcal{A_i}$.  The intersection of all closed intervals is denoted $\bigcap \mathcal{A}$. It is defined as
\begin{align*}
	\bigcap \mathcal{A} = \{x: x \in A \text{ } \forall A \in \mathcal{A}\}.
\end{align*}
In other words, we are looking for some set $A$ such that $A$ is a subset of every other set in $\mathcal{A}$. This set is $A = \{x: 0 < x < 1\}$. Consider any other subset $C$ of $\mathcal{A}$. If $C \neq A$, then necessarily there must exist an element $x$ such that $x \in C$ and $x \notin A$, showing that $A \subset C$ but $C \nsubseteq A$. Since $x$ is not in every subset of $\mathcal{A}$, $x \notin \bigcap \mathcal{A}$. 
\end{question}

\begin{question}{1.1.6}
What is the union of all of the closed intervals contained in the open
interval $(0, 1)$? Justify your answer. \\

Let $\mathcal{A}$ be the set containing all sets containing $(0,1)$ as a subset. The union of all these sets is denoted by $\bigcup \mathcal{A}$. An object $x$ is an element of $\bigcup \mathcal{A}$ if there exists some set $C \subset \mathcal{A}$ such that $x \in C$. We need to consider two cases: either the object $x \leq 0$ or $ 1 \leq x$. The case of $0 < x < 1$ is trivial. For any $x$ such that $1\leq x$ we can create a set $C$ such that $C = \{ y: 0 < y < x \}$. This since $1 \leq x$, it is guaranteed that $C \subset \mathcal{A}$. The case of $x \leq 0$ is analogous. Hence, $\bigcup \mathcal{A} = (- \infty, \infty)$.
\end{question}

\begin{problem}{1.1.7}
If A is a collection of subsets of a set X, formulate and prove a theorem
like Theorem 1.1.5 (\emph{from book numbering}) for the intersection and union of A. 

\begin{theorem}{1.1.7}
	Let $\mathcal{A}$ be a collection of subsets $A_1, A_2,...,A_n$ of some set $X$. Then $( \bigcup \mathcal{A})^c = A_1^c \cap A_2^c \cap ... \cap A_n^c$ and $( \bigcap \mathcal{A})^c = A_1^c  \cup A_2^c \cup ... \cup A_n^c$.
\end{theorem}

\begin{proof}
This is a generalization of DeMorgan's law, proved in the book. We begin with the statement $( \bigcup \mathcal{A})^c = A_1^c \cap A_2^c \cap ... \cap A_n^c$.	We can rewrite $ (\bigcup \mathcal{A})^c$ as $( A_1 \cup A_2 \cup ... \cup A_n)^c$. We can then sub-partition this collection of unions into a collection of two unions, as such:
\begin{align*}
	(\bigcup \mathcal{A})^c = [ A_1 \cup ( A_2 \cup ... \cup A_n) ]^c
\end{align*}
Then we will refer to $A_2 \cup ... \cup A_n$ as $B$. We can then rewrite the above as $(A_1 \cup B)^c$, for which DeMorgans laws apply. Thus, we write $(A_1 \cup B)^c =  A_1^c \cap B^c = A_1^c \cap ( A_2 \cup ... \cup A_n)^c$. As next step, we sub partition B into two sets, as such
\begin{align*}
	( A_2 \cup ... \cup A_n)^c = [ A_2 \cup (A_3 \cup .... \cup A_n)]^c
\end{align*}
Then DeMorgan's laws apply again as above, and we can write $[ A_2 \cup (A_3 \cup .... \cup A_n)]^c = A_2^c \cap (A_3 \cup ... \cup A_n)^c$. Since intersections and unions are associative, we can then write
\begin{align*}
	(\bigcup \mathcal{A})^c = (A_1^c \cap (A_2^c \cap (A_3 \cup ... \cup A_n)^c)) = A_1^c \cap A_2^c \cap (A_3 \cup ... \cup A_n)^c	
\end{align*}
We continue an inductive application of DeMorgan's laws as outlined above, until we see that $( \bigcup \mathcal{A})^c = A_1^c \cap A_2^c \cap ... \cap A_n^c$ \\~\\
The other proof is analogous, requiring a sub-partition of the collection of intersections and rewriting them into series of intersections of two sets to which DeMorgan's laws apply.
\end{proof}

\end{problem}

\begin{problem}{1.1.8}
Which of the following functions $f : \mathbb{R} \to \mathbb{R}$ are one to one and which ones are onto. Justify your answer. \\
(a) $f (x) = x^2$ ; This function is neither onto, nor one-to-one. It is not onto, since there is no $x$ such that $f(x) < 0$. It is not one-to-one since $f(x)=f(-x)$ for all $x \in \mathbb{R}$. \\
(b) $f (x) = x^3$ ; This function is both one-to-one and onto. It is one-to-one since there $f(x) \neq f(y)$ for all $x,y$ such that $x \neq y$. It is onto, as for any $y \in \mathbb{R}$, there exists an $x \in \mathbb{R}$ such that $f(x) = y$. \\
(c) $f (x) = e^x$ This function is one-to-one, but not onto. It is one-to-one, for $f(x) \neq f(y)$ for all $x,y \in \mathbb{R}$ such that $x \neq y$. It fails to be onto since there exists no $x$ such that $f(x)<0$ for any $x \in \mathbb{R}$.
\end{problem}

\begin{theorem}{1.1.9}
If $f : A \to B$ is a function and $E$ and $F$ are subsets of $B$, then $f^{-1} (E \cap F ) = f^{-1} (E) \cap f^{-1} (F )$.
\end{theorem}

\begin{proof}
If $x \in f^{-1} (E \cap F )$, then $f(x) \in E \cap F$. This means that $f(x)$ is both in $E$ as well as in $F$. If $f(x) \in E$, then $x \in f^{-1}(E)$. If $f(x) \in F$, then $x \in f^{-1}(F)$. Since $f(x)$ is in both $E$ and $F$, $x$ is in $f^{-1} (E \cap F)$. \\
Assume $x$ is in $f^{-1} (E) \cap f^{-1} (F )$. Then, $x \in f^{-1} (E)$ as well as $x \in f^{-1} (F)$. If $x \in f^{-1} (E)$, then $f(x) \in E$. If $x \in f^{-1}(F)$, then $f(x) \in F$. Since $x$ is both in $f^{-1} (E)$ as well as $f^{-1} (F)$, we know that $f(x) \in E \cap F$. This implies that $x \in f^{-1} (E \cap F)$. \\ 
Since every $x \in f^{-1} (E \cap F )$ implies that $x \in f^{-1} (E) \cap f^{-1} (F )$ and vice versa, it is true that  $f^{-1} (E \cap F ) = f^{-1} (E) \cap f^{-1} (F )$.
\end{proof}

\begin{theorem}{1.1.10}
If $f : A \to B$ is a function and $E$ and $F$ are subsets of $B$, then $f^{-1} (E \backslash F ) = f^{-1} (E) \backslash f^{-1} (F )$ if $F \subset E$.
\end{theorem}

\begin{proof}
If $x \in f^{-1} (E \backslash F )$, then $f(x) \in E \backslash F$. Thus $f(x) \in E$ but $f(x) \notin F$. This means that $x \in f^{-1}(E)$ and but also $x \notin f^{-1}(F)$. In other words, $x \in f^{-1}(E) \backslash f^{-1} (F)$. \\
Assume now that $x \in f^{-1}(E) \backslash f^{-1} (F)$. Then $x \in f^{-1}(E)$ but $x \notin f^{-1}(F)$. This means that $f(x) \in E \backslash F$, and hence $x \in f^{-1} (E \backslash F)$. \\
It follows that  $f^{-1} (E) \backslash f^{-1} (F ) = f^{-1} (E \backslash F)$.
\end{proof}

\begin{theorem}{1.1.11}
If $f : A \to B$ is a function and $E$ and $F$ are subsets of $A$, then $f (E \cup F ) = f (E) \cup f (F )$.
\end{theorem}

\begin{proof}
If $y \in f(E \cup F)$, then $y = f(x)$ for some $x \in E$ or $x \in F$. If $x \in E$, then $y \in f(E)$. If $x \in F$, then $y \in f(F)$. Since $x$ is in either one of these, we know that $y \in f(E) \cup f(F)$. \\
Assume now that $y \in f(E) \cup f(F)$. This implies that $y=f(x)$ for some $x \in E$ or $x\in F$. Thus we can write $x \in E \cup F$. Then $y \in f(E \cup F)$. \\
Since any element of $f(E \cup F)$ is in $f(E) \cup f(F)$ and vice versa, we conclude that $f (E \cup F ) = f (E) \cup f (F )$.
\end{proof}


\begin{theorem}{1.1.12}
If $f : A \to B$ is a function and $E$ and $F$ are subsets of $A$, then $f (E \cap F ) \subset f (E) \cap f (F )$.
\end{theorem}

\begin{proof}
Assume that $y \in f(E \cap F)$. Then $y = f(x)$ for some $x \in E \cap F$. This means that both $x \in E$ as well as $x \in F$. Then, $f(x) \in f(E)$ and $f(x) \in f(F)$, showing that $f(x) \in f(E) \cap f(F)$, or - equivalently - that $y \in f(E) \cap f(F)$. This proves that $f (E \cap F ) \subset f (E) \cap f (F )$.
\end{proof}

\begin{question}{1.1.13}
Give an example of a function $f : A \to B$ and subsets $F\subset E$ of $A$ for
which $f (E) \backslash f (F ) = f (E \backslash F )$. \\

The above conditions are fulfilled for a function $f(x)=x$ with $A=B=[0,10]$, and the subsets $E=[1,6]$ and $F = [1,2] \subset E$.
\end{question}

\begin{exercise}{1.1.14}
\emph{Solved in Class}
\end{exercise}

\begin{exercise}{1.1.15}
\emph{Solved in Class}
\end{exercise}

%%%%%%%%%%%%%%%%%%%%%%%%%%%%%%%%%%%%%%%%%
\begin{center}
\textsc{\Large Section 1.4}
\end{center}
%%%%%%%%%%%%%%%%%%%%%%%%%%%%%%%%%%%%%%%%%


\begin{exercise}{1.4.1}
For each of the following sets, describe the set of all upper bounds for the set: \\
(a) the set of odd integers; The integers are unbounded.\\
(b) $\{1 - 1/n : n \in \mathbb{N}\}$; The set of all upper bounds for this set is $\{x \in \mathbb{N}: x \geq 1\}$. \\
(c) $\{r \in \mathbb{Q}: r^3 < 8\}$; The set of all upper bounds for this set is $\{x \in \mathbb{Q}: x \geq 2\}$.\\
(d) $\{ \sin x : x \in \mathbb{R}\}$; The set of all upper bounds for this set is $\{x \in \mathbb{R}: x \geq 1\}$.
\end{exercise}



\begin{exercise}{1.4.2}
For each of the sets in (a), (b), (c) of the preceding exercise, find the least upper bound of the set, if it exists.\\

(a) There is no upper bound, and hence no least upper bound. \\
(b) The least upper bound is $1$.\\
(c) The least upper bound is $2$.
\end{exercise}



\begin{theorem}{1.4.3}
If a subset $A$ of $\mathbb{R}$ is bounded above, then the set of all upper bounds for $A$ is a set of the form $[x, \infty)$. \emph{What is $x$}?
\end{theorem}

\begin{proof}
Let $B$ denote the set of all upper bounds of $A$. By definition, a number $m \in \mathbb{R}$ is considered an upper bound for the set $A$ if $z \leq m$ for all $z \in A$. If the set $A$ has a largest number, then this largest number - $y'$ - will be in the set $B$. In that case, it is obvious that all numbers $m > y'$ will also be upper bounds, since we assumed that $x \leq y'$ for all $x \in A$, and that $m > y'$, it follows that $x \leq y' < m$. Therefore, the set $[y', \infty)$ would be the set of all upper bounds of $A$. \\
Assume now that $A$ does not have a largest number. By the completeness theorem we know that any subset $A$ of an ordered field - such as $\mathbb{R}$ - is indeed bounded above. Specifically, according to theorem 1.4.4 of the book we know that any subset of $\mathbb{R}$ not only is bounded above, but has a least upper bound. By definition, a number $c$ is a least upper bound if and only if it is a number such that $x \leq c$ for all $x \in A$, and for every $k \in \mathbb{R}$, if $k$ is an upper bound of $A$, then $k \geq c$. It is obvious then that the set of all upper bounds of $A$ will be the set $[c, \infty)$ where $c$ is the least upper bound of $A$.
\end{proof}


\begin{exercise}{1.4.4}
\emph{Solved in Class}
\end{exercise}

\begin{exercise}{1.4.7}
\emph{Solved in Class}
\end{exercise}

%%%%%%%%%%%%%%%%%%%%%%%%%%%%%%%%%%%%%%%%%%%%%
\begin{center}
\textsc{\Large Section 1.5}
\end{center}
%%%%%%%%%%%%%%%%%%%%%%%%%%%%%%%%%%%%%%%%%%%%%

\begin{exercise}{1.5.1}
For each of the following sets, find the set of all extended real numbers $x$ that are greater than or equal to every element of the set. Then find the sup of the set. Does the set have a maximum? \\
(a) $(-10, 10)$; The set of all numbers greater than this set is the set $[10, +\infty)$. The supremum of the set in question is $10$. The set does not have a maximum.\\
(b) $\{n^2 : n \in \mathbb{N}\}$; In the extended set of real numbers, the only element greater than or equal to all the elements in the set in question is $+\infty$, which thereby must also be its supremum. The set does not have a maximum.\\
(c) $\{ \frac{2n+1}{n+1}\}$; The set of all real numbers greater than the set in question is the set $[2,\infty)$. The supremum is $2$ and the set does not have a maximum.
\end{exercise}



\begin{exercise}{1.5.2}
Find the sup and inf of the following sets. Tell whether each set has a maximum or a minimum. \\
(a) $(-2,8]$; The infimum of the set is $-2$ and the supremum is $8$. The has a maximum, but not a minimum. \\
(b) $\frac{n+2}{n^2+1}$; The infimum of the set is 0, and the supremum is $2$. The set has a maximum, but no minimum.\\
(c) $\{ n/m : n,m \in \mathbb{Z}, n^2<5m^2\}$; The infimum of the set is $- \sqrt{5}$, and the maximum is $\sqrt{5}$. Seeing that $\sqrt{5}$ is not a rational number, the set has neither a maximum nor a minimum.
\end{exercise}




\begin{exercise}{1.5.3}
Prove that if $\sup A < \infty$, then for each $n \in \mathbb{N}$ there is an element $a_n \in A$
such that $\sup A - 1/n < a_n \leq \sup A$. 

	\begin{proof}
	This is true since we can easily construct an element $a_n$ such that this equality holds. We assume that $A$ is defined for all $m/n$ with $m,n \in \mathbb{Z}$ within $A$. In this case, we constructs our term to be $a_n = \sup A - 1/(n+1)$. It is obvious that since $1/(n+1) < 1/n$, that $\sup A - 1/n < \sup A - 1/(n+1) \leq \sup A$. \\
	Alternatively, we may also note that $\sup A - 1/n < \sup A$ for all $n \in \mathbb{N}$ by definition, so the inequality holds in the trivial case of $a_n = \sup A$. 
	\end{proof}
\end{exercise}




\begin{exercise}{1.5.4}
Prove that if $\sup A = \infty$, then for each $n \in \mathbb{N}$ there is an element $a_n \in A$
such that $a_n > n$. 

\begin{proof}
Assume some set $A$ whose supremum is $+\infty$. In that case, $\forall x \in A$, $x < \infty$. Both from the Archimedean property and from the Peano Axioms we know that for every $n \in \mathbb{N}$, there is a successor element $n'$ which is also in $\mathbb{N}$, such that $n < n'$. Since there $\nexists a$ such that $a = \infty$, and $n<\infty$, this implies that $\exists a_n$ such that $a_n = n'$ and $a_n \in A$, showing that $n < a_n < \infty$.
\end{proof}
\end{exercise}




\begin{exercise}{1.5.5}
Formulate and prove the analog of Theorem 1.5.4 for $\inf$. \\

\textbf{Theorem.} Let $A$ be a non-empty subset of $\mathbb{R}$ and let $x$ be an element of $\mathbb{R}$. Then \\ 
(a) $\inf A \geq x$ if and only if $a \geq x$ for every $a \in A$;\\
(b) $x > \inf A$ if and only if $x > a$ for some some $a \in A$.

	\begin{proof}
	By definition, $a \geq x$ if and only if $x$ is a lower bound for $A$. If $x$ is a lower bound for $A$, then $A$ is bounded below. This implies that its $\inf$ is its greatest lower bound, which is necessarily greater than or equal to $x$. Conversely, if $\inf A \geq x$, then $\inf A$ is finite and is the greatest lower bound for $A$. Since $\inf A \geq x$, $x$ is also  a lower bound for $A$. Thus, $\inf A \geq x$ if and only if $a \geq x$ for every $a \in A$. \\
	If $x > \inf A$, then $x$ is not a lower bound for $A$, which means $x >a$ for some $a \in A$. Conversely, if $x > a$ for some $a  \in A$, then $x> \inf A$, since $a \geq \inf A$. Thus, $x > \inf A$ if and only if $x > a$ for some $a \in A$.
	\end{proof}
\end{exercise}




\begin{exercise}{1.5.6}
Prove part (d) of Theorem 1.5.7.\\

\textbf{Theorem}. Let $A,B$ be non-empty subsets of $\mathbb{R}$. Then $\sup (A-B) = \sup A - \inf B$.

	\begin{proof}
	According to the book, $\sup(A+B) = \sup A + \sup B$ (proof on p. 30). We can then write $\sup (A + (-B) ) = \sup A + \sup(-B)$. We then apply Theorem 1.5.7b, to rewrite $\sup(-B)$ as $-\inf A$. From this it follows that 
	\begin{align*}
	\sup(A + (-B)) = \sup(A-B) = \sup A + (- \inf B) = \sup A - \inf B
	\end{align*}.
	\end{proof}
\end{exercise}




\begin{exercise}{1.5.7}
Prove (e) of Theorem 1.5.7. \\

\textbf{Theorem}. Let $A,B$ be non-empty subsets of $\mathbb{R}$. If $A \subset B$, then $\sup A \leq \sup B$ and $\inf B \leq \inf A$.
	\begin{proof}
	If $A \subset B$, then $a \in A$ implies that $a \in B$ for all $a$. Then, if $\sup A \in A$, $\sup A \in B$. Since $\sup B \geq b$ for all $b \in B$, it is obvious that $\sup A \leq \sup B$. Assume now that $\sup A \notin A$. In that case, $\sup A - \epsilon \in A$ for all $\epsilon > 0$. Thus, $\sup A - \epsilon \in A$ and $\sup A - \epsilon \in B$. By definition, $\sup B$ is greater than or equal to all $b \in B$. This means  that if $\sup A - \epsilon \in B$ implies that $\sup A \leq \sup B$. The proof for the infimum is analogous.
	\end{proof}
\end{exercise}

\begin{exercise}{1.5.10}
Prove (a) of Theorem 1.5.10. \\

\textbf{Theorem}. Let $f$ and $g$ be functions defined on a set containing $A$ as a subset, and let $c \in \mathbb{R}$ be a positive constant. Then $\sup_A c f= c \sup_A f$ and $\inf_A c f = c \inf_A f$.
	\begin{proof}
	Let $f$ be function $f: A \to B$. Then $\sup f$ is the supremum of $B$ provided that $f$ is surjective. Let $M$ be an arbitrary upper bound of $cx$ for some $x \in B$. We say that $c x \leq M$ if and only if $x \leq M/c$. This shows that $M$ is an upper bound of $c x$ if and only if $M/c$ is an upper bound of $B$. Hence, $\sup c B = c \sup B$ and similarily $\sup c f = c \sup f$. The result for the infimum follows similarily.
	\end{proof}
\end{exercise}

\begin{exercise}{1.5.8}
\emph{Solved in Class}
\end{exercise}

\begin{exercise}{1.5.9}
\emph{Solved in Class}
\end{exercise}

\begin{exercise}{1.5.11}
Prove (b) of Theorem 1.5.10. \\

\textbf{Theorem}. Let $f$ and $g$ be functions defined on a set containing $A$ as a subset, and let $c \in \mathbb{R}$ be a positive constant. Then $\sup_A (-f) = - \inf_A f$.

	\begin{proof}
	We have a function $f: A \to B$. A number $x$ is a lower bound for $f(a)$ for all $a \in A$ if and only if $-x$ is an upper bound for the set $-f(a)$. Let  $L$ be the set of all lower bounds for $f(a)$. Then $-L$ is the set of all upper bounds for $-f(a)$. Furthermore, the largest member of $L$ and the smallest member of $-L$ are negatives of each other. That is, $- \inf f(a) = \sup (f(a))$, or equivalently $- \inf f = \sup (-f)$.  
	\end{proof}
\end{exercise}

\begin{exercise}{1.5.12}
Prove (c) of Theorem 1.5.10.\\

\textbf{Theorem}. Let $f$ and $g$ be functions defined on a set containing $A$ as a subset, and let $c \in \mathbb{R}$ be a positive constant. Then $\sup_A (f+g) \leq \sup_A f + \sup_A g $ and $\inf_A f + \inf_A g \leq \inf_A (f+g)$.

	\begin{proof}
	By definition, $f(a) \leq \sup f$ for all $a \in A$ and $g(a) \leq \sup g$ for all $a \in A$. Therefore, $f(a)+g(a) \leq \sup f + \sup g$. Let $c$ denote the supremum of $f+g$. We know that $\sup f + \sup g$ is an upper bound  for $f(a)+g(a)$. Since the supremum is always less than or equal to an upper bound, we find that $c \leq \sup f + \sup g$. This implies that $\sup (f+g) \leq \sup f + \sup g$.
	\end{proof}
\end{exercise}

\begin{exercise}{1.5.13}
Prove (d) of Theorem 1.5.10. \\

\textbf{Theorem}. Let $f$ and $g$ be functions defined on a set containing $A$ as a subset, and let $c \in \mathbb{R}$ be a positive constant. Then $\sup \{ f(x) - f(y): x, y \in A \} = \sup_A f - \inf_A f$.
	\begin{proof}
	This appears somewhat obvious. The function $f$ is defined on $A$, i.e., for every $a \in A$, $f$ maps to some value $f(a)$ in some set, let's call it $B$. The value $\sup f$ is defined as to be the least upper bound of $f(a)$, i.e. $\nexists x$ such that $f(x) > \sup f$ for some $x \in A$. The infimum is defined as the value such that there is no value $x \in A$ such that $x < \inf f$. The value defined by $f(x) - f(y)$ for all $x,y \in A$ is a measure of the distance between these two values. Since $\sup f$ and $\inf f$ are defined as above, we can see that there cannot be a greater distance between any other two points in $B$ than the distance between $\sup f$ and $\inf f$. Therefore, for any collection of distances between points in $B$ reached by $f(x)$ for all points $x \in A$, the supremum of this collection - namely, the largest value of this set such that no other value is larger - cannot be any other than the distance between the supremum and the infimum of the function itself.
	\end{proof}
\end{exercise}


%%%%%%%%%%%%%%%%%%%%%%%%%%%%%%%%%%%%
\begin{center}
\textsc{\Large Section 2.1}
\end{center}
%%%%%%%%%%%%%%%%%%%%%%%%%%%%%%%%%%%%

\begin{exercise}{2.1.1}
Show that \\
\textbf{(a) if $|x - 5| < 1$, then $x$ is a number greater than $4$ and less than $6$.;} 
This is equivalent to saying $-1 < x - 5 < 1$. We add $5$ to the inequality, and we get $4 < x < 6$.\\
\textbf{(b) if $|x - 3| < 1/2$ and $|y - 3| < 1/2$, then $|x - y| < 1$;} We add the inequalities, such that we see $ |x-3| + |y - 3| < 1/2 + 1/2 = 1$. We notice that $|y-3| = |3-y|$. We rewrite using the triangle inequality:
	\begin{align*}
	|(x-3)+(3-y)| \leq |x-3| + |3-y| &< 1 \\
	|x - y| \leq |x-3| + |3-y| &< 1.
	\end{align*}
\textbf{(c) if $|x - a| < 1/2$ and $|y - b| < 1/2$, then $|x + y - (a + b)| < 1$.} We add the inequalities and get $|x-a| + |y-b| < 1/2 + 1/2 =1$. We can then rewrite using the triangle inequality as above
	\begin{align*}
		|(x-a) + (y-b) | &\leq |x-a| + |y-b| & < 1 \\
		|x+y - a-b| &\leq |x-a|+ |y-b| &<1\\
		|x+y - (a-b)|  &\leq |x-a|+ |y-b| &<1.
	\end{align*}
\end{exercise}

\begin{exercise}{2.1.3}
Put each of the following sequences in the form $a_1, a_2, a_3, \hdots, a_n$. This requires that you compute the first 3 terms and find an expression for the $n$th term. \\
\textbf{(a) the sequence of positive odd integers;} This is a sequence of the form $1,3,5,\hdots$. To find the $n-th$ term, we express this sequence as $a_n = 2 n - 1$, with $n \in \N$. \\
\textbf{(b) the sequence defined inductively by $a_1 = 1$ and $a_{n+1} = - \frac{a_n}{2}$;} The sequence begins with $1, -1/2, 1/4, \hdots$. The $n$th term will be something like $a_n= ((-1)^{n-1})/(2^{n-1})$ for $n \in \N$. \\
\textbf{(c) the sequence defined inductively by $a_1 = 1$ and $a_{n+1} = \frac{a_n}{n+1}$}. This is the series $1, 1/3, 1/12, 1/60, \hdots$. The $n$th term is: $a_n = \frac{2}{(n+1)!}$.
\end{exercise}

\begin{exercise}{2.1.4}
Find $\lim 1/n^2$. \\

The larger $n$ become, the smaller $1/n^2$ will become. We guess the limit to be $0$. For any $\epsilon > 0$, we need an $N$ such that whenever $n > N$, $1/n^2 < \epsilon$. We find that this is true whenever $ 1/\epsilon < n^2$, or in other words - whenever $\sqrt{1/ \epsilon} < n$.
\end{exercise}


\begin{exercise}{2.1.5}
Find $\lim  \frac{2n-1}{3n+1}$. \\

We guess the limit to be $2/3$. 
	\begin{align*}
	|\frac{2n-1}{3n+1} - \frac{2}{3}| = | \frac{3(2n-1) - 2(3n+1)}{3(3n+1)}| = | \frac{6n-3-6n-2}{9n+3} |\\
	= |\frac{5}{9n+3} < | \frac{5}{9n} | < | \frac{5}{n} |
	\end{align*}
We must choose an $n > N$ such that $N> \frac{5}{\epsilon}$ so that this will be true.
\end{exercise}


\begin{exercise}{2.1.6}
Find $\lim (-1)^n /n$ \\

We guess the limit to be $0$. We see $| \frac{(-1)^n}{n}| = |\frac{1}{n}|$. Hence we need to choose an $n>N$ such that $N> \frac{1}{\epsilon}$ for this inequality to be true.
\end{exercise}

\begin{exercise}{2.1.9}
\emph{Solved in Class}
\end{exercise}


\begin{exercise}{2.1.10}
Prove that $\lim 2^{-n} = 0$. Hint: prove first that $2^n \geq n$ for all natural numbers $n$.\\

	\begin{proof} 
	We wish to show that $2^n > n$ for all $n$. Proof by induction. The base case, $2^1 > 1$ is obviously true, since $2^1 = 2$. We assume now that $2^n > n$ for some $n$. Then we wish to check $2^{n+1}$. But, we can rewrite this simply as $2^n 2^1$. Let $k= 2^n$. Since we know that $k > n$, it is obvious that $2 k > n+1$. Thus, $2^n > n$ for all $n \in \N$. \\
	We note that $2^{-n} = \frac{1}{2^n}$. Thus, $\lim 2^{-n}  = \lim \frac{1}{2^n}$. Since $2^n$ increases until infinity, we see that $1/2^{n}$ will grow smaller and smaller, since $1/2^{n} > 1/2^{n+1}$ for all $n$.\\
	We see that for any $\epsilon > 0$, we need to simply pick $n$ such that $1/\epsilon < 2^n$.  As such, the limit is 0.
	\end{proof}
\end{exercise}



\begin{exercise}{2.1.11}
Prove that if $a_n \to 0$ and $k$ is any constant, then $k a_n \to 0$. \\

If $a_n \to 0$, this means that $a_n < \epsilon$ for any $\epsilon > 0$. We multiply by $k$ and find that $k a_n < k \epsilon$.
\end{exercise}



%%%%%%%%%%%%%%%%%%%%%%%%%%%%%%%
\begin{center}
\textsc{\Large Section 2.2}
\end{center}
%%%%%%%%%%%%%%%%%%%%%%%%%%%%%%


\begin{exercise}{2.2.1}
Make an educated guess as to what you think the limit is, then use the definition of limit to prove that your guess is correct.\\

$\lim \frac{3n^2 -2}{n^2 +1}$. I assume the limit will be $3$. We note that $\frac{3n^2 - 2}{n^2+1} <  \frac{3n^2}{n^2} = 3$. Hence, the limit is 3.
\end{exercise}

\begin{exercise}{2.2.2}
\emph{Solved in Class}
\end{exercise}

\begin{exercise}{2.2.3}
Make an educated guess as to what you think the limit is, then use the definition of limit to prove that your guess is correct.\\

$\lim \frac{1}{\sqrt{n}} $ I assume the limit will be $0$. We see $| \frac{1}{\sqrt{n}}| = | \frac{1}{n^{1/2}}|$; therefore, this is true whenever we choose an $n>N$ such that $\sqrt{N}> \frac{1}{\epsilon}$.
\end{exercise}

\begin{exercise}{2.2.4}
\emph{Solved in Class}
\end{exercise}


\begin{exercise}{2.2.5}
Make an educated guess as to what you think the limit is, then use the definition of limit to prove that your guess is correct.\\

$\lim (\sqrt{n^2 + N} -n) $ I know as we approach infinity, the limit is $1/2$, but have not been able to prove it.
\end{exercise}


\begin{exercise}{2.2.6}
Make an educated guess as to what you think the limit is, then use the definition of limit to prove that your guess is correct.\\

$\lim (1 + 1/n)^3 = 1$. Proof:
	\begin{align*}
		| (1+ \frac{1}{n})^3 -1 | &= | \frac{1}{n^3} + \frac{3}{n^2}+ \frac{3}{n} + 1 - 1| = | \frac{1}{n^3} + \frac{3}{n^2}+ \frac{3}{n} |	
		\end{align*}
We note that each term is of the form $c/n$ or multiples thereof for some constant $c$. It has already been shown that each such term tends can be made smaller than any $\epsilon$. This also holds for the sum.
\end{exercise}


\begin{exercise}{2.2.8}
Prove that if $\lim a_n = a$, then $\lim a_n^3 = a^3$. 
	\begin{align*}
	|a_n^3 - a^3 | = |(a_n - a) (a_n^2 + a_n a + a^2)|
	\end{align*}
We then note that we are given that $|a_n - a| < \epsilon$. From this we see that 
	\begin{align*}
	|(a_n - a) (a_n^2 + a_n a + a^2)| < \epsilon (a_n^2 + a_n a + a^2).
	\end{align*}
\end{exercise}


\begin{exercise}{2.2.9}
Does the sequence $\{cos(n \pi /3)\}$ have a limit? Justify your answer. \\

No. The sequence $\{cos(n \pi /3)\}$ oscillates between $-1$ and $1$; a limit cannot converge to two different values. Hence, this sequence does not have a limit.
\end{exercise}

\begin{exercise}{2.2.10}
\emph{Solved in Class}
\end{exercise}

\begin{exercise}{2.2.11}
Prove that if $\{a_n\}$ and $\{b_n\}$ are sequences with $|a_n| \leq b_n$ for all $n$ and if $\lim b_n =0$, then $\lim a_n = 0$ also. \\

We are given that $ |a_n|  \leq b_n$ for all $n$. Therefore, we know that $ \lim |a_n| \leq \lim b_n$. We know that $\lim b_n = 0$. Hence we can write - equivalently - that $\lim |a_n| \leq 0$. We notice that $|a_n|$ is defined to be greater than or equal to zero. Hence we have $0 \leq \lim |a_n| \leq 0$, from which it follows by the squeeze theorem (proof on p. 43 of the book) $\lim |a_n| = 0$.
\end{exercise}


\begin{exercise}{2.2.12}
Prove the following partial converse to Theorem 2.2.3: Suppose $\{a_n\}$ is a convergent sequence. If there is an $N$ such that $a_n \leq c$ for all $n > N$, then $\lim a_n \leq c$. Also, if there is an $N$ such that $b \leq a_n$ for all $n > N$, then $b \leq \lim a_n$. \\

Note that ${a_n}$ is bounded by $c$ according to the premise. In this case, we can say that $a_n \leq \sup a_n \leq c$ for all $n$. Let $ a = \lim a_n$. We know by definition that $a \leq \sup a_n$, and therefore we can write that $\lim a_n \leq \sup a_n \leq c$. \\
Likewise, we can say that $b$ is a lower bound for $a_n$ such that $b \leq \inf a_n$. We know that by definition $\inf a_n \leq a$, allowing us to write $b \leq \lim a_n$.
\end{exercise}


%%%%%%%%%%%%%%%%%%%%%%%%%%%%%%%%%%%%%
\begin{center}
\textsc{\Large Section 2.3}
\end{center}
%%%%%%%%%%%%%%%%%%%%%%%%%%%%%%%%%%%%%

\begin{exercise}{2.3.1}
\emph{Solved in Class}
\end{exercise}


\begin{exercise}{2.3.2}
Use the Main Limit Theorem to find $\lim \frac{n^2 - 5}{n^3 + 2n^2 + 5}$. \\
	\begin{align*}
		\lim \frac{n^2 - 5}{n^3 + 2n^2 + 5} &=  &(\text{dividing top and bottom by $n^3$}) \\
		\lim \frac{1/n - 5/n^3}{1+2/n + 5/n^3} &= \\
		\frac{\lim (1/n - 5/n^3)}{\lim (1+2/n + 5/n^3)} &= \frac{0}{1} = 0.
	\end{align*}
\end{exercise}

\begin{exercise}{2.3.3}
\emph{Solved in Class}
\end{exercise}

\begin{exercise}{2.3.4}
\emph{Solved in Class}
\end{exercise}

\begin{exercise}{2.3.5}
Prove Theorem 2.3.2.

	\begin{proof}
	We know that $\lim a_n = 0$, hence we know that for all $\epsilon > 0$, there exists an $N$ such that whenever $n>N$, $|a_n | < \epsilon$. Likewise, we know that $b_n$ is bounded, such that we can state that $- q \leq b_b \leq q$. We can then also write $|a_n| < \frac{\epsilon}{|q|}$.  We guess that the limit of $ (a_n)(b_n)$ is zero, so we write:
		\begin{align*}
		|a_n b_n - 0| = |a_n b_n| &\leq | a_n q| \\
		|a_n b_n| &\leq |q| \frac{\epsilon}{|q|}  \\
		|a_n b_n| &\leq \epsilon.
		\end{align*}
	Thus, the $\lim a_n b_n = 0$, since we there is an $N$ such that the above inequality is true whenever we pick an $n>N$.
	\end{proof}
\end{exercise}


\begin{exercise}{2.3.6}
Prove that a sequence $\{a_n\}$ is both bounded above and bounded below if and only if its sequence of absolute values $\{|a_n|\}$ is bounded above. 

\begin{proof}
By definition, if $\{|a_n|\}$ is bounded above, then there exists some $M$ such that $|a_n| \leq M$ for all $n$. This is equivalent to saying $-M \leq a_n \leq M$, which proves that $\{a_n\}$ is bounded above and below.
\end{proof}
\end{exercise}


\begin{exercise}{2.3.7}
Prove part(b) of Theorem 2.3.6. 

\begin{proof}
	Since both $a_n$ and $b_n$ have a limit, we can write $|a_n -a| < \frac{\epsilon}{2}$ and $|b_n - b| < \frac{\epsilon}{2}$. For all $\epsilon$, we have an $N$ such that if we choose $n>N$, these inequalities are true. We know add them together and find
	\begin{align*}
	|a_n - a| + |b_n - b| &< \epsilon \\
	|(a_n) + (b_n -b) | \leq |a_n - a| + |b_n - b| &< \epsilon \\
	|(a_n + b_n) - (a+b) | \leq |a_n - a| + |b_n - b| &< \epsilon.
	\end{align*} 
\end{proof}
\end{exercise}


\begin{exercise}{2.3.8}
Prove that if $\{b_n\}$ is a sequence of positive terms and $b_n \to b > 0$, then there is a number $m>0$ such that $b_n \geq m$ for all $n$. \\

This is true by virtue of the definition of $\mathbb{R}$. The statement above is equivalent to saying that we are looking for some $m$ such that $0 < m \leq b_n$. By definition $\mathbb{R}$ is full, such that between any two numbers, there are infinitely more numbers.
\end{exercise}


\begin{exercise}{2.3.9}
Prove part (d) of Theorem 2.3.6. Hint: Use the previous exercise. I.e, that if $a_n \to a$ and $b_n \to b$, $a_n/b_n \to a/b$, if $b \neq 0$ and $b_n \neq 0$ for all $n$. 

	\begin{proof}
		\begin{align*}
		|a_n \frac{1}{b_n} - a \frac{1}{b}| = |a_n \frac{1}{b_n} - a \frac{1}{b_n} + a \frac{1}{b_n} - a \frac{1}{b} | \leq |a_n = a| |\frac{1}{b_n}| + |a| |\frac{1}{b_n -b}|
		\end{align*}
		We know that $\{1/b_n\}$ is bounded, and hence $\{|1/b_n|\}$ is bounded above. We also have $|a_n -a| \to 0$. Therefore, $|a_n -a| |1/b_n| \to 0$. Also, $ |a| |1/b_n - 1/b| \to 0$. By (b) we know that $ |a_n - a| |1/b_n| + |a| |1/b_n - 1/b| \to 0$, proving that $a_n/b_n \to a/b$.
	\end{proof}
\end{exercise}


\begin{exercise}{2.3.10}
Prove part (f) of theorem 2.3.6. Hint: use the identity:
	\begin{align*}
	x^k - y^k = (x-y)(x^{k-1} + x^{k-2} y+ \hdots  + y^{k-1})
	\end{align*}
with $x= a_n^{1/k}$ and $y= a^{1/k}$, to show that $a_n^{1/k} \to a^{1/k}$ if $a_n \geq 0$ for all $n$. 

	\begin{proof}
		We notice that 
		\begin{align*}
			a_n^{1/k} - a^{1/K} = (a_n - a) (a_n^{1/K-1} + a_n^{1/K -2}a^{1/k} + \hdots + a^{1/K-1}) = (a_n - a) b_n
		\end{align*}
		where
		\[ b_n = a_n^{1/K-1} + a_n^{1/K -2}a^{1/k} + \hdots + a^{1/K-1} \]
		We know that $\{a_n\}$ converges, and hence that $\{|a_n|\}$ is bounded above. We choose an upper bound $m$ for $\{|a_n|\}$ which satisfies that $|a_n| \leq m$. Then $b_n \leq \frac{1/k} m^{1/k}$ sowing that $\{|b_n|\}$ is bounded above. According to theorem 2.3.2 we conclude that $|a_n - a| |b_n| \to -$ and find from theorem 2.3.1 that $a_n^{1/k} \to a^{1/k}$.
	\end{proof}
\end{exercise}


\begin{exercise}{2.3.12}
Prove that $\lim a^{1/n} = 1$. Hint: use the result of the previous exercise.\\

We notice that $n^{1/n} > 1$ for all $n \in \N$.We can therefore write that we are looking for a solution to $n^{1/n} - 1 < \epsilon$.We can rearrange and raise both sides to the $n$th power, resulting in the equation $n < (1 + \epsilon)^n$. We can expand the right hand side using the binomial theorem:
	\begin{align*}
	n < 1+ n \epsilon + \frac{1}{2}n(n-1) \epsilon^2 + \hdots
	\end{align*}
As long as $n < \frac{1}{2} n(n-1) \epsilon^2$ this inequality holds, requiring that $n> 1+ \frac{2}{\epsilon^2}$. Therefore, for any $\epsilon >0$ there exists an $N$ such that whenever $n>N$,  $| n^{1/n} -1|< \epsilon$.
\end{exercise}


%%%%%%%%%%%%%%%%%%%%%%%%%%%%%%%%%%%%%
\begin{center}
\textsc{\Large Section 2.4}
\end{center}
%%%%%%%%%%%%%%%%%%%%%%%%%%%%%%%%%%%%%

\begin{exercise}{2.4.1}
Tell which of the following sequences are non-increasing, non-decreasing, bounded? Justify your answers.\\
(a) $\{n^2\}$; for $n \in \N$, this sequence is non-decreasing since $n^2 < (n+1)^2$ for all $n$. It is bounded below by $1$.\\
(b) $\{ \frac{1}{\sqrt{n}} \}$; this sequence is non-increasing, since $\sqrt{n}=n^{1/2}<(n+1)^{1/2}$ for all $n$. This then implies $\frac{1}{\sqrt{n}} > \frac{1}{\sqrt{n+1}}$. The sequence is bounded by $0$ and $1$. \\
(c) $\{ \frac{(-1)^n}{n} \}$;  this sequence is neither non-increasing, nor non-decreasing as the sign of the value of the sequence fluctuates due to the term $(-1)^n$. It is, however, bounded by $-1$ and $1/2$.\\
(d) $\{ \frac{n}{2^n} \}$; this is the sequence $\frac{1}{2},\frac{2}{4},\frac{3}{8},\hdots$ which is clearly non-increasing. It is bounded by $0$ and $1$. \\
(e) $\{ \frac{n}{n+1} \}$; this is the sequence $\frac{1}{2}, \frac{2}{3},\frac{3}{4},\frac{4}{5},\hdots$ which is clearly non-decreasing and tending to $1$. It is bounded by $1/2$ and $1$.
\end{exercise}


\begin{exercise}{2.4.2}
	Prove that the sequence ${x_n}$ with $x_1=1$ and $x_{n+1} = \sqrt{x_n + 1}$ converges and decide what number it converges to.
	\begin{proof}
		The first few terms of the sequence: $1, \sqrt{2}, \sqrt{1+\sqrt{2}}, \sqrt{1+\sqrt{1+\sqrt{2}}}, \hdots$. We see that $x_n$ is both increasing an bounded:
			\[ x_{n+1} = \underbrace{\sqrt{1+\sqrt{1+ \hdots \sqrt{1+\sqrt{2}}}}}_\text{$n$ terms} > \underbrace{\sqrt{1+\sqrt{1+ \hdots \sqrt{1+\sqrt{2}}}}}_\text{$n-1$ terms} = x_n \]
			Then $x_n$ is increasing for all $n \in \N$. We prove that $x_n < 2$ by induction. We have $x_1=1<2$. Then, if $x_k<2$, $x_{k+1} = \sqrt{1+x_k}< \sqrt{1+2}= \sqrt{3}<2$. Thus by the monotone convergence theorem, $x_n$ converges and is bounded by 2. \\
		Finding the limit: We solve $a=\sqrt{1+a}$, which implies $a^2 - a -1 =0$. The solutions to this are $a_1 = \frac{1+\sqrt{5}}{2}$, $a_2= \frac{1-\sqrt{5}}{2}$. The correct limit is $a_1$, since $a_2 < 0$.
	\end{proof}		
\end{exercise}

\begin{exercise}{2.4.3}
If $a_1 = 1$ and $a_{n+1} = (1-2^{-n})a_n$ , prove that $\{a_n\}$ converges.
	\begin{proof}
We notice that $2^{-n}$ is monotone and converges to $0$. Therefore we see that $1-2^{-n}$ is also monotone, converging to $1$. The whole term then is monotone and non-increasing. It is also bounded by 0 and 1. Therefore, by the monotone convergence theorem, $a_n$ converges.
	\end{proof}
\end{exercise}


\begin{exercise}{2.4.4}
	
\end{exercise}


\begin{exercise}{2.4.8}
Prove that $\lim \frac{n^5 + 3n^3 + 2}{n^4 -n+1}= \infty$. 
	\begin{align*}
	\lim \frac{n^5 + 3n^3 + 2}{n^4 -n+1} &= \lim \frac{n^5(1+3n^3/n^5+2/n^5)}{n^4(1-n/n^4+1/n^4)}\\
	&= \lim \underbrace{n}_\text{$\to \infty$} (\underbrace{\frac{1+3/n^2+2/n^5}{1-1/n^3+1/n^4}}_\text{$\to 1$})
	\end{align*}
And hence, $\lim \frac{n^5 + 3n^3 + 2}{n^4 -n+1}= \infty$.
\end{exercise}


\begin{exercise}{2.4.11}
Prove Part (c) of Theorem 2.4.7. \\
 	\textbf{Theorem} $\lim a_n = \infty$ iff $\lim(-a_n)= -\infty $
 	
 	\begin{proof}
 	If $a_n \to \infty$, there exists some value of $a_n$ such that $a_n > M$ for any possible $M\in\R$. If we consider the sequence $a_n (-1)$, we see clearly that $-a_n < M$ for any $M\in\R$. But if that is true, then $\lim -a_n= -\infty$.
 	\end{proof}
\end{exercise}

\begin{exercise}{2.4.14}
\emph{Solved in Class}
\end{exercise}



%%%%%%%%%%%%%%%%%%%%%%%%%%%%%%%%%%%%%
\begin{center}
\textsc{\Large Section 2.5}
\end{center}
%%%%%%%%%%%%%%%%%%%%%%%%%%%%%%%%%%%%%

\begin{exercise}{2.5.1}
Give an example of a nested sequence of bounded open intervals that does not have a point in its intersection.
\end{exercise}

\begin{exercise}{2.5.4}
Prove by induction that if $\{n_k\}$ is an increasing sequence of natural numbers, then $n_k \geq k$ for all $k$.
	\begin{proof}
	Assume the base case $n_k=n$, which is the series $1,2,3,4,5,\hdots$. Since $k$ is the counter index, i.e. $k \in \N$, it is obvious that $n_k = k = 1,2,3,4,5, \hdots$. We generalize to the $n+1$ case, i.e. $n_k = n+1$. In that case we have $n_k = n+1 = k+1 > k$.
	\end{proof}
\end{exercise}


\begin{exercise}{2.5.5}
Which of the following sequences $\{a_n \}$ have a convergent subsequence? Justify your answer. \\
(a) $a_n= (-2)^n$; None of the subsequences are convergent, as they either tend to $+\infty$ or $-\infty$.\\
(b) $a_n = \frac{5+(-1)^n n}{2+3n}$; This sequence is convergent for all $n$ such that $n\mod 2=0$, which is the sequence starting with $0.875, 0.6428, 0.55, 0.5, 0.46875, 0.4473, 0.4318, 0.42, 0.41071, \hdots$\\
(c) $a_n=2^{(-1)^n}$ This sequence has convergent subsequences for all $n$ such that $n\mod 2=0$ and for $n \mod 2 = 1$.
\end{exercise}

\begin{exercise}{2.5.7}
For each of the following sequences, determine how many different limits of subsequences there are. Justify your answer.\\
(a) $\{1+(-1)^n\}$; This sequence is $0,2,0,2,0,2,\hdots$ and as such has two different limits: 0 and 2. \\
(b) $\{cos(n \pi/3)\}$; There are two different limits. The first approaches $1$ for the sequence of all $n$ where $n\mod 6=0$. The second limit is attained at $-1$ for all $n$ such that $n\mod 6=3$. \\
(c) $1, \frac{1}{2},1,\frac{1}{2}, \frac{1}{3},1, \frac{1}{2},\frac{1}{3},\frac{1}{4},1. \frac{1}{2},\frac{1}{3},\frac{1}{4}, \frac{1}{5},\hdots$ The terms $a_1, a_3, a_6, a_{10}, a_15, \hdots$ are convergent to 1.
\end{exercise}

\begin{exercise}{2.5.8}
Does the sequence $\sin n$ have a convergent subsequence? Why? \\

Yes, it has three convergent subsequences, provided $n\in \R$. If $n\in \N$, then it is not convergent. 
\end{exercise}

\begin{exercise}{2.5.9}
Prove that a sequence which satisfies $|a_{n+1} - a_n| < 2^{-n}$ for all $n$ is a Cauchy sequence.

	\begin{proof}
	We notice that the sequences defined by the above condition are non-increasing and covergent. We notice the following pattern:
		\begin{align*}
		|a_{n+2}-a_n| = |a_{n+2}-a_{n+1}+a_{n+1}-a_n| &\leq |a_{n+2} - a_{n+1}|+ |a_{n+1} - a_n| < 2^{-n+1} + 2^{-n} \\
		|a_{n+3}-a_n| = |a_{n+3}-a_{n+2}+a_{n+2}-a_{n+1}+a_{n+1}-a_n| &\leq |a_{n+3}-a_{n+2}|+|a_{n+2}-a_{n+1}|+|a_{n+1}-a_n|\\
		&< 2^{-(n+2)} + 2^{-(n+1)} + 2^{-n}
		\end{align*}
	Inductively, we see that this pattern continues for all patterns $a_n$ and $a_n+k$ with $k \in \N$. Now, we assume two indices $m$ and $n$ such that $m>n$. We find
		\begin{align*}
		|a_m - a_n| &< 2^{-(m-1)} + 2^{-(m-2)}+\hdots+2^{-n} = \\
				& 2^{-n} \underbrace{ (1+ 2^{-1}+2^{-2}+\hdots+2^{-(m-1)+n})}_\text{geometric series}
		\end{align*}
	We rewrite and solve the geometric series:
		\begin{align*}
		2^{-n}( \sum_{k=-m+n+1}^0 2^k)= 2^{-n}(2-2^{-m+n+1}) = 2^{1-n} - 2^{1-m} 
		\end{align*}
	We want to to prove $|2^{1-n} - 2^{1-m}| \leq 2^{1-n} + 2^{1-m} < \epsilon$. We solve the equations $2^{1-n} < \frac{\epsilon}{2}$ and $2^{1-m} < \frac{\epsilon}{2}$. The solution to this is $2- \frac{\log(\epsilon)}{\log(2)}< n,m$
	\end{proof}
\end{exercise}


\begin{exercise}{2.5.10}
\emph{Solved in class}
\end{exercise}

%%%%%%%%%%%%%%%%%%%%%%%%%%%%%%%%%%%%%
\begin{center}
\textsc{\Large Section 2.6}
\end{center}
%%%%%%%%%%%%%%%%%%%%%%%%%%%%%%%%%%%%%

\begin{exercise}{2.6.1}
\emph{Solved in Class}
\end{exercise}

\begin{exercise}{2.6.2}
Find lim inf and lim sup for the sequence $a_n = \frac{n}{2^{k_n}}-1$ with $k_n$ being the largest integer $k$ so that $2^k \leq n$. \\

This is the sequence $0, 0, \frac{1}{2},0, \frac{1}{4}, \frac{2}{4}, \frac{3}{4}, 0, \frac{1}{8}, \frac{2}{8}, \frac{3}{8}, \frac{4}{8}, \frac{5}{8}, \frac{6}{8}, \frac{7}{8}, 0, \hdots $. It is clear that $\lim \inf =0$ and $\lim \sup = 1$. 
\end{exercise}


\begin{exercise}{2.6.3}
Find lim inf and lim sup for the sequence $1, \frac{1}{2}, 1,\frac{1}{2},\frac{1}{3},1,\frac{1}{2},\frac{1}{3}, \frac{1}{4},1,\hdots$. \\

We find that $\lim \inf = 0$ and $\lim \sup =1$.
\end{exercise}


\begin{exercise}{2.6.4}
\emph{Solved in class}
\end{exercise}

\begin{exercise}{2.6.5}
If $\lim \sup a_n$ is finite, prove that $\lim \inf(-a_n) = - \lim \sup a_n$.
	
	\begin{proof}
	By assumption, $\lim \sup a_n$ is equal to some $a$, such that $a \geq a_n$ for all $a_n \in \{a_n\}$. We multiply this by $-1$ to find the inverse sequence $\{-a_n\}$. Then we have $-a \leq -a_n$ for all $a_n \in \{a_n\}$. By definition, this means $-a = \lim \inf (-a_n)$. Therefore, $\lim \inf (-a_n) = - \lim \sup a_n$.
	\end{proof}
\end{exercise}


\begin{exercise}{2.6.7}
\emph{Solved in Class}
\end{exercise}

\begin{exercise}{2.6.8}
If $\{a_n\}$ and $\{b_n\}$ are non-negative sequences and $\{b_n\}$ converges, prove that $\lim \sup a_n b_n= (\lim \sup a_n) (\lim b_n)$.

	\begin{proof}
	We need to consider two cases. First, assume $\{a_n\}$ is \emph{not} bounded above. Then $\lim \sup a_n  = \infty$. It then doesn't matter what we multply $a_n$ with, we will always get infinity provided that $b_n \neq 0$.Then $\lim \sup (a_n b_n) = \lim \sup a_n \lim b_n = \infty$. \\
	We now consider case 2, where $a_n$ is bounded above. By Bolzano-Weierstrass we know that $a_n$ then has at least one convergent subsequence. Let $a$ be the subsequential limit of $a_n$, and let $M$ be the upper bound of $a_n$. We know then that $\lim \sup a_n$ exists and $\lim \sup a_n \leq M$. \emph{MORE WORK NEEDED ON THIS}. We note that according to the main limit theorem, if $a_n \to a$ and $b_n \to b$, $a_n b_n \to a b$. Thus $\lim \sup (a_n b_n) = \lim \sup a_n \lim b_n$.
	\end{proof}
\end{exercise}


\begin{exercise}{2.6.9}
\emph{Solved in class}
\end{exercise}


\begin{exercise}{2.6.12}
Which numbers do you think are subsequential limits of $\{ \sin n\}_{n=1}^\infty$? Can you prove that your guess is correct? \\

All $x \in R$ with $|x| \leq 1$ are limits for $\sin$.
\end{exercise}




%%%%%%%%%%%%%%%%%%%%%%%%%%%%%%%%%%%%%
\begin{center}
\textsc{\Large Section 3.1}
\end{center}
%%%%%%%%%%%%%%%%%%%%%%%%%%%%%%%%%%%%%

\begin{exercise}{3.1.1}
    If $f$ is a function with domain $[0,1]$, what is the domain of $f(x^2 -1)$? \\
    
    $g$ is defined at point $x$ iff $x^2-1 \in [0,1]$, $0 \leq x^2-1 \leq 1$.
    \begin{align*}
        \begin{cases}
            x^2-1 & x\in (-\infty,-1] \cup [1,\infty] \\
            x^2 \leq 2 & x \in [-\sqrt{2},\sqrt{2}]
         \end{cases}
    \end{align*}
    Thus $x \in [-\sqrt{2},-1] \cup [1,\sqrt{2}]$.
\end{exercise}

\begin{exercise}{3.1.2}
What is the natural domain of the function $\frac{x^2 +1}{x^2-1}$? With this as its domain, is this function continuous? Why? \\

The domain is $\R \backslash \{-1,1\}$. The function is continuous everywhere except for the points not part of the domain.
\end{exercise}

\begin{exercise}{3.1.4}
Show that the function $f(x) = |x|$ is continuous on all of $\R$.

	\begin{proof}
	We need to find a $\delta$ such that for any $\epsilon > 0$, we have $||x| - |a|| < \epsilon$ whenever $|x-a|< \delta$.
	\end{proof}
\end{exercise}

\begin{exercise}{3.1.5}
Assuming $\sin$ is continous, prove that $\sin(x^3-4x)$ is continuous.

	\begin{proof}
	We know that $|sin(x)| < 1$ for all $x$.  
	\end{proof}
\end{exercise}

\begin{exercise}{3.1.8}
    We know $\sqrt{x}$ is continuous at all $a \geq 0$ by theorem 3.1.7. Give another proof of this fact by using only the definition of continuity. 
    
    \begin{proof}
    We need to distinguish between two cases: \\
    Case 1 - $a= 0$: $|\sqrt{x} - \sqrt{0}| = \sqrt{x} < \epsilon$ iff $0 \leq x < \epsilon^2$, $\delta=\epsilon^2$. Whenever $x<\epsilon^2$ we find that $\sqrt{x}< \epsilon$ and therefore $\sqrt{x}$ is continuous at $a=0$. \\
    Case 2 - $a>0$: $|x-a|=|\sqrt{x}-\sqrt{a}| |\sqrt{x}+\sqrt{a}|$. This implies $|\sqrt{x}-\sqrt{a}| = \frac{x-a}{\sqrt{x}-\sqrt{a}} \leq \frac{|x-a|}{\sqrt{a}} < \epsilon$ if we have $|x-a| < \epsilon \sqrt{a}$, $\delta=\epsilon \sqrt{a}$.
    \end{proof}
\end{exercise}

\begin{exercise}{3.1.9}
Consider the function:
	\begin{align*}
	f(x) = \left\{
     \begin{array}{lr}
       1 & : x \geq 0,\\
       -1 & : x < 0
     \end{array}
   \right.
	\end{align*}
Is this function continuous if its domain is $\R$? Is it continous if its domain is cut down to $\{ x\in \R: x \geq 0 \}$? How about if its domain is $\{x \in \R : x \leq 0 \}$?
\end{exercise}

\begin{exercise}{3.1.10}
Let $f$ be a function with domain $D$ and suppose $f$ is continuous at some point $a \in D$. Prove that, for each $\epsilon >0$, there is a $\delta >0$ such that
	\begin{align*}
	|f(x) - f(y) | < \epsilon \text{ whenever } x,y \in D \cap (a-\delta,a+ \delta)
	\end{align*}
\end{exercise}

\begin{exercise}{3.1.11}
    Prove that the function
        \[ f(x)= \begin{cases} \sin (1/x) & x\neq 0 \\ 0 & x =0 \end{cases} \]
    is not continuous at 0.
    
    \begin{proof}
    Whenever $x_n \to 0$ we have $f(x_n) \to f(0) = 0$. We are looking for a sequence $x_n \to 0$ but where $f(x_n) \not\to f(0)=0$. We choose $x_n = \frac{1}{\pi/2 + 2 \pi n}$. This goes to 0 but $\sin(\frac{\pi}{1} +2n)=1$.
    \end{proof}
\end{exercise}



\begin{exercise}{3.1.12}
    Prove that the function
        \[ f(x)= \begin{cases} x \sin(1/x) & x \neq 0 \\ 0 & x=0 \end{cases} \]
    is continuous at 0.
    
    \begin{proof}
    We need to estimate $|f(x)-f(0)|$.
    \[ |f(x)-f(0)| = |x \sin \left(\frac{1}{x} \right)| = |x| |\sin \left( \frac{1}{x} \right)| \leq |x| < \epsilon \]
    Thus $|x-0|<\epsilon$ for $\delta = \epsilon$.
    \end{proof}
\end{exercise}



%%%%%%%%%%%%%%%%%%%%%%%%%%%%%%%%%%%%%
\begin{center}
\textsc{\Large Section 3.2}
\end{center}
%%%%%%%%%%%%%%%%%%%%%%%%%%%%%%%%%%%%%


\begin{exercise}{3.2.2}
    Prove that if $f$ is a continuous function on a closed bounded interval $I$ and if $f(x)$ is never 0 for $x \in I$, then there is a number $m>0$ such that $f(x)\geq m$ for all $x \in I$ or $f(x) \leq -m$ for all $x \in I$.
    
    \begin{proof}
    Assume $f(a)>0$. We have that $f([a,b])=[m,M]$. We know that $m=\min f$, $M=\max f$. Let's prove that $m>0$. By contradiction: assume $m<0$. Value 0 is taken \emph{(non-legible)} $a_n$ intermediate value $[m, f(a)]$ which contradicts $f(x)\neq 0$. Prove for case 2 is analogous (show that $M <0$).
    \end{proof}
\end{exercise}



\begin{exercise}{3.2.3}
    Prove that if $f$ is a continuous function on a closed bounded interval $[a,b]$ and if $(x_0, y_0)$ is any point in the plane, then there is a closest point to $(x_0,y_0)$ on the graph of $f$. 
    
    \begin{proof}
    Pick any point $x \in [a,b]$. Then the distance to $x_0,y_0$ is $\text{dist}\left( (x_0,y_o), (x,f(x)) \right) = \left( (x-x_0)^2 + (y_0-f(x))^2)\right)^{\frac{1}{2}}$. We must prove that this function attains its minimum value in $[a,b]$ and that if $f$ is continuous, then $\left( (x-x_0)^2 + (y_0-f(x))^2)\right)^{\frac{1}{2}}$ is also continuous on $[a,b]$. Then distance takes its minimum value there.
    \end{proof}
\end{exercise}


\begin{exercise}{3.2.4}
Find an example of a function which is continuous on a bounded (but not closed) interval $I$, but is not bounded. Then find an example of a function which is continuous and bounded on a bounded interval $I$, but does not have a maximum value. \\

The function $f: (0,1) \to \R$ with $f(x)=\frac{1}{x}$ fulfills the first condition. \\
The second condition cannot be fulfilled; according to theorem 3.2.1 (p. 65): ``If $f$ is a continuous function on a closed bounded interval $I$, then $f$ is bounded on $I$ and in fact, it assumes both a minimum and a maximum value on $I$.'' The only way to create a function which would \emph{not} assume a maximum on such an interval would be by violating the continuity. For example, the function	
	\begin{align*}
	f(x) =
		\begin{cases}
		2x & x< 1/2 \\
		0  & x \geq 1/2
		\end{cases}
	\end{align*}
fails to achieve its maximum on a bounded interval $[0,1]$. However, it does so by having a discontinuity at $x = 1/2$.
\end{exercise}



\begin{exercise}{3.2.7}
Give an example of a function defined on the interval $[0,1]$ which does not take on every value between $f(0)$ and $f(1)$. \\

In other words, we are looking for a function with a discontinuity between $[0,1]$. One example would be:
 \begin{displaymath}
   f(x) = \left\{
     \begin{array}{lr}
       x & : 0 \leq x < \frac{1}{2} \\
       2x & : \frac{1}{2} \leq x \leq 1
     \end{array}
   \right.
\end{displaymath} 

\end{exercise}


\begin{exercise}{3.2.8}
    Show that if $f$ and $g$ are continuous functions on the interval $[a,b]$ such that $f(a)< g(a)$ and $g(b)<f(b)$, then there is a number $c \in (a,b)$ such that $f(c)=g(c)$. 
    
    \begin{proof}
    We create a function $h(x)=f(x)-g(x)$. This is continuous since it is a linear combination of continuous functions, and it is defined on $[a,b]$. We know that $h(a)=f(a)-g(a)<0$ and $h(b)=f(b)-g(b)>0$.  Bt the intermediate value theorem there exists a $c$ such that $h(c)=f(c)-g(c)=0$, which implies $f(c)=g(c)$.
    \end{proof}
\end{exercise}



\begin{exercise}{3.2.9}
Let $f$ be a continuous function from $[0,1]$ to $[0,1]$.Prove that there is a point $c \in [0,1]$ such that $f(c) = c$ - that is, show that $f$ has a \emph{fixed point}. Hint: Apply the Intermediate Value Theorem to the function $g(x) = f(x)-x$. 
	\begin{proof}
	Let $g(x) = f(x) -x$. Since $f(x)$ is continuous, we know that $g(x)$ is also continuous. Then $g(a) \geq 0$ and $g(b) \leq 0$. By the intermediate value theorem we know that there exists some $x \in [0,1]$ such that $g(x) = 0$, which implies that $f(x)=x$.
	\end{proof}
\end{exercise}


\begin{exercise}{3.2.10}
    Use the intermediate value theorem to prove that if $n$ is a natural number, then every positive number $a$ has a positive $n$-th root.
    
    \begin{proof}
    We write the function $f(x)=x^n$ which is continuous on $[0,\infty)$ since it is a polynomial. We notice $f(0)=0<a$. We know that there is a number $m\in \N$ such that $m>a$ which implies $f(m)=m^n \geq m > a$. Thus we have $f(0) < a$ and $f(m)>a$ and since $f$ is continuous on $[0,m]$, the intermediate value theorem states that there exists a $c$ such that $f(c)=c^n=a$.
    \end{proof}
\end{exercise}


\begin{exercise}{3.2.11}
Prove that a polynomial of odd degree has at least one real root. 

	\begin{proof}
	Assume $g(x): \R \to \R$ is an odd degree polynomial. Then $g$ is of the form $\sum_{k=0}^{n} a_k x^k$, where $a_k$ is the $k$-th coefficient of the polynomial an $n \in \N$ such that $n \mod 2 = 1$. We can then factor $g$ to be of the form $g(x)= x^n (a_n + \sum_{k=0}^{n-1} a_k \frac{x^k}{x^n} )$. We note that $\lim_{x \to \pm \infty} \sum_{k=0}^{n-1} a_k \frac{x^k}{x^n}  = 0$. We then consider $\lim_{x \to \pm \infty} x^n a_n$. We note that since $n$ is odd, $x^n \leq 0$ if $x \leq 0$ and $x^n \geq 0$ if $x \geq 0$. Therefore, $\lim_{x \to +\infty} x^n a_n = + \infty$ and $\lim_{x \to -\infty} x^n a_n = -\infty$, provided that $a_n > 0$. Hence, we find that $\lim_{x \to - \infty} g(x) = - \infty$ and $\lim_{x \to \infty} g(x) = \infty$. In the case of $a_n < 0$, we find that $\lim_{x \to -\infty} g(x) = \infty$ and $\lim_{x \to \infty} g(x)= -\infty$. \\
	We know that any polynomial is continous, and the above shows that there are some $a,b \in \R$ such that $g(a) < 0$ and $g(b) > 0$. We now consider the interval $[a,b]$. By the Intermediate value theorem, we find that for every $c \in [g(a),g(b)]$ there exists some $x \in [a,b]$ such that $g(x)=c$, implying that there exists at least one $x$ such that $g(x)=0$. 
	\end{proof}
\end{exercise}



\begin{exercise}{3.2.12}
Use the Intermediate Value Theorem to prove that $f$ is a continuous function on an interval $[a,b]$ and if $f(x) \leq m$ for every $x \in [a,b)$, then $f(b) \leq m$.
	\begin{proof}
	Assume that $m < f(b)$, such that $f(x) \leq m < f(b)$ for all $x \in [a,b)$. We then know that $m = f(b)- \delta$ for some $\delta > 0 \in \R$. But, by properties of the real numbers, we would also have $m = f(b) - \delta < f(b) - \epsilon < f(b)$ some some $\epsilon$, such as $\epsilon = \frac{\delta}{2}$. But $f(b)- \epsilon \in [a,b)$ - contradiction: by the intermediate value theorem, since $f$ is continuous, we know that there exists some $x$ such that $f(x)= f(b) - \epsilon$, and thus we require $m \geq f(b) - \epsilon$. Hence, $f(b) \leq m$.
	\end{proof}
\end{exercise}



%%%%%%%%%%%%%%%%%%%%%%%%%%%%%%%%%%%%%
\begin{center}
\textsc{\Large Section 3.3}
\end{center}
%%%%%%%%%%%%%%%%%%%%%%%%%%%%%%%%%%%%%

\begin{exercise}{3.3.1}
Is the function $f(x)=x^2$ uniformly continuous on $(0,1)$? Justify your answer. \\

Yes, it is. According to Theorem 3.3.4, if a function is continuous on a closed bounded interval $I$, it is uniformly continuous there. Assume $I=[0,1]$. Then by theorem 3.3.4, $f$ is uniformly continuous on $I$. By theorem 3.3.6, $f$ is then also uniformly continuous on $(0,1)$.
\end{exercise}


\begin{exercise}{3.3.2}
    Is the function $f(x)=1/x^2$ uniformly continuous on $(0,+\infty)$ (\emph{actual text printed asks only about interval up to 1})? Justify your answer. \\
    
    Assume $f$ were uniformly continuous. Then it it is uniformly continuous on a subinterval, such as $(0,1)$. But $f(x)$ is \emph{not} bounded on the interval $(0,1)$. Therefore, it is not uniformly continuous. 
\end{exercise}


\begin{exercise}{3.3.3}
Is the function $f(x) = x^2$ uniformly continuous on $(0, + \infty)$? Justify your answer.  \\

No, it its not. As $x \to \infty$ we find that the distance between $y,y'$ gets bigger and bigger, such that $x,x'$ need to be closer and closer for $y,y'$ to still be within $\epsilon$ of each other. This means that $\delta$ does depend on $a$, so it is not uniformly continuous. 
\end{exercise}


\begin{exercise}{3.3.4}
    Using only the $\epsilon-\delta$ definition of uniform continuity, prove that the function $f(x)=\frac{x}{x+1}$ is uniformly continuous on $[0,\infty)$.
    
    \begin{proof}
        \begin{align*}
        |f(x)-f(y)| &= | \frac{x}{x+1} - \frac{y}{y+1} | = |\frac{x(y+1) + y(x+1)}{(x+1)(y+1)}| \\
         &= |\frac{xy+x-xy-y}{(x+1)(y+1)}| = \frac{|x-y|}{(x+1)(y+1)} \leq |x-y|
         \end{align*}
    Estimate $|f(x)-f(y)|\leq |x-y|$. Then for all $\epsilon> 0$, $\delta = \epsilon$ implies $|x-y|<\delta = \epsilon$ will result in $|f(x)-f(y)|<\epsilon$.
    \end{proof}
\end{exercise}



\begin{exercise}{3.3.5}
    In example 3.3.8 we showed that $\sqrt{x}$ is uniformly continuous on $[1,\infty)$. Show that it is also uniformly continuous on $[0,1]$. \\
    
    By theorem 3.3.4: if $\sqrt{x}$ is continuous on $[0,1]$, it is uniformly continuous there.
\end{exercise}


\begin{exercise}{3.3.6}
Prove that if $I$ and $J$ are overlapping intervals in $\R (I \cap J \neq \emptyset$ and $f$ is a function, defined on $I \cup J$, which is uniformly continuous on $I$ and uniformly continuous on $J$, then it is also uniformly continuous on $I \cup J$. Use this and the previous exercise to prove that $\sqrt{x}$ is uniformly continuous on $[0, + \infty)$. 

	\begin{proof}
	By assumption $I \cap J \neq \emptyset$. We shall assume that the interval $I$ is the ``lower'' one of the two. Then there exists an $x$ such that $x \in I \cap J$. Since $I$ is uniformly continuous by assumption, we know that $[x-a,x]$ is uniformly continous for all $(x-a)\in I$. Likewise we know that $[x,x+b]$ is uniformly continous for all $(x+b)\in J$ since $J$ is uniformly continous by assumption. This implies that the whole interval $[x-a, x+b]$ is uniformly continous. \\
	To prove that $\sqrt{x}$ is uniformly continuous, we assume we are given some $\epsilon >0$. For this, we pick $\delta=\epsilon^2$. We note that $|\sqrt{x} - \sqrt{y}| \leq |\sqrt{x} + \sqrt{y}|$. If $|x-y|< \delta = \epsilon^2$, we find:
		\begin{align*}
		|\sqrt{x} - \sqrt{y}|^2 \leq |\sqrt{x} - \sqrt{y}| |\sqrt{x}+\sqrt{y}| = |x-y| < \epsilon^2
		\end{align*}
	This guarantees that $|\sqrt{x} - \sqrt{y}|<\epsilon$, proving that $\sqrt{x}$ is uniformly continous on $(0,\infty)$.
	\end{proof}
\end{exercise}



\begin{exercise}{3.3.8}
Let $f$ be a function defined on an interval $I$ and suppose that there are positive constants $K$ and $r$ such that 
	\[ |f(x)-f(y)| \leq K |x-y|^r \text{ for all } x,y \in I. \]
Prove that $f$ is uniformly continuous.

	\begin{proof}
	According to assumption, we find that $|f(x)-f(y)| \leq K |x-y|^r$ for all $x,y \in I$. This implies that if $K | x-y|^r < \epsilon$, $|f(x)-f(y)|<\epsilon$. Thus we find that we need to solve $K |x-y|^r < \delta \leq \epsilon$, and find that for any given $\epsilon$, we pick a $\delta$ such that $\delta = \sqrt[r]{\frac{\epsilon}{K}}$. Since $\delta$ does not depend on where $x,y$ are in the interval, $f$ is uniformly continuous.
	\end{proof}
\end{exercise}



\begin{exercise}{3.3.9}
Is the function $f(x)=\sin(\frac{1}{x})$ continuous on $(0,1)$? Is it uniformly continuous on $(0,1)$? Justify your answers. 
	
	\begin{proof}
	Since $\sin$ is a trigonometric function, it is continuous on its whole domain. Likewise, $\sin(1/x)$ is continous since it is merely a composition of two elementary functions. \\
	However, $\sin(1/x)$ is \emph{not} uniformly continous. The reason for this is that the as $x \to 0$ the function oscillates between $-1$ and $1$. Thus, a $\delta$ that would work at one point in the function will can produce potentially a difference $|f(x)-f(y)|=2$ for $x,y$ sufficiently close to 0. Hence, the functions is not uniformly continous.
	\end{proof}
\end{exercise}


\begin{exercise}{3.3.10}
    Is the function $f(x)=x \sin(1/x)$ uniformly continuous on $(0,1)$? Justify your answer. \\
    
    \begin{proof}
    Method 1: $f(1)=\sin(1)$. It is still uniformly continuous. $\lim_{x \to 0} f(x)= \lim_{x \to 0} x \sin(1/x) = 0$. By squeeze theorem:
		\begin{align*}
		\underbrace{0}_\text{$\to$ 0} \leq \overbrace{|x \sin \left( \frac{1}{x} \right) |}^\text{$\to$ 0 by squeeze thrm.} \leq \underbrace{|x|}_\text{$\to$ 0}
		\end{align*}		       
         
    If we define $f(0)=0$, $f(1)=\sin(1)$, then $f(x)$ becomes continuous on $[0,1]$. then by theorem 3.3.4, $f$ is uniformly continuous on $[0,1] \implies f$ is uniformly continuous on $(0,1)$.\\
    Method 2: 
        \[ |f(x)-f(y)| = |x \sin(1/x) - y \sin(1/y)| \leq |x \sin(1/x)| + |y \sin(1/y)| \leq |x|+|y| \]
    Then for all $\epsilon>0$ we have $|f(x)-f(y)|<\epsilon$ if $x,y \in (0,\frac{\epsilon}{2}]$. If now $x,y > \epsilon/3$, then there exists a $\delta>0$ such that $|f(x)-f(y)| <\epsilon$ whenever $|x-y| < \delta$: 
        \[ |f(x)-f(y)| \leq |x|+|y| < \frac{\epsilon}{3} + |x| + |y-x| < \frac{\epsilon}{3} + \frac{\epsilon}{3}+ \delta < \epsilon \]
    if $\delta< \frac{\epsilon}{3}$. Then we choose $\delta= \min(\frac{\epsilon}{3}, \delta_0)$.
    \end{proof}
\end{exercise}


%%%%%%%%%%%%%%%%%%%%%%%%%%%%%%%%%%%%%
\begin{center}
\textsc{\Large Section 3.4}
\end{center}
%%%%%%%%%%%%%%%%%%%%%%%%%%%%%%%%%%%%%




%%%%%%%%%%%%%%%%%%%%%%%%%%%%%%%%%%%%%
\begin{center}
\textsc{\Large Section 4.1}
\end{center}
%%%%%%%%%%%%%%%%%%%%%%%%%%%%%%%%%%%%%



%%%%%%%%%%%%%%%%%%%%%%%%%%%%%%%%%%%%%
\begin{center}
\textsc{\Large Section 4.2}
\end{center}
%%%%%%%%%%%%%%%%%%%%%%%%%%%%%%%%%%%%%



%%%%%%%%%%%%%%%%%%%%%%%%%%%%%%%%%%%%%
\begin{center}
\textsc{\Large Section 4.3}
\end{center}
%%%%%%%%%%%%%%%%%%%%%%%%%%%%%%%%%%%%%

\begin{exercise}{4.3.1}
\end{exercise}


\begin{exercise}{4.3.2}
\end{exercise}

\begin{exercise}{4.3.3}
\end{exercise}

\begin{exercise}{4.3.4}
\end{exercise}

\begin{exercise}{4.3.5}
\end{exercise}

\begin{exercise}{4.3.6}
\end{exercise}

\begin{exercise}{4.3.7}
\end{exercise}


\begin{exercise}{4.3.8}
\end{exercise}


\begin{exercise}{4.3.9}
\end{exercise}


\begin{exercise}{4.3.10}
\end{exercise}


\begin{exercise}{4.3.11}
\end{exercise}

\begin{exercise}{4.3.12}
\end{exercise}


\begin{exercise}{4.3.13}
\end{exercise}


\begin{exercise}{4.3.14}
\end{exercise}

\begin{exercise}{4.3.15}
\end{exercise}

\begin{exercise}{4.3.16}
\end{exercise}

%%%%%%%%%%%%%%%%%%%%%%%%%%%%%%%%%%%%%
\begin{center}
\textsc{\Large Section 4.4}
\end{center}
%%%%%%%%%%%%%%%%%%%%%%%%%%%%%%%%%%%%%


\begin{exercise}{4.4.1}
\end{exercise}


\begin{exercise}{4.4.2}
\end{exercise}



\begin{exercise}{4.4.3}
\end{exercise}



\begin{exercise}{4.4.4}
\end{exercise}




\begin{exercise}{4.4.5}
\end{exercise}



\begin{exercise}{4.4.6}
    Find the limit $\lim_{x \to \infty} \frac{\ln x}{x^r}$ where $r > 0$. \\
    
    This limit is indeterminate. Use L'H\^{o}pital's rule. We find $\frac{d}{dx} \ln x = 1/x$ and $\frac{d}{dx} x^r = \frac{r\, x^r}{x}$.
\end{exercise}



\begin{exercise}{4.4.7}
\end{exercise}


\begin{exercise}{4.4.8}
\end{exercise}


\begin{exercise}{4.4.9}
\end{exercise}


\begin{exercise}{4.4.10}
\end{exercise}


\begin{exercise}{4.4.11}
\end{exercise}

\begin{exercise}{4.4.12}
\end{exercise}


\begin{exercise}{4.4.13}
\end{exercise}


\begin{exercise}{4.4.14}
\end{exercise}

\begin{exercise}{4.4.15}
\end{exercise}

\begin{exercise}{4.4.16}
\end{exercise}

%%%%%%%%%%%%%%%%%%%%%%%%%%%%%%%%%%%%%
\begin{center}
\textsc{\Large Section 5.1}
\end{center}
%%%%%%%%%%%%%%%%%%%%%%%%%%%%%%%%%%%%%





% --------------------------------------------------------------
%     You don't have to mess with anything below this line.
% --------------------------------------------------------------

\end{document}

